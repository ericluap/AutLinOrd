% In this file you should put the actual content of the blueprint.
% It will be used both by the web and the print version.
% It should *not* include the \begin{document}
%
% If you want to split the blueprint content into several files then
% the current file can be a simple sequence of \input. Otherwise It
% can start with a \section or \chapter for instance.
\chapter{Bumps, blocks, and bubbles}
\section*{Bubbles}
Our goal here is to define the bubble relation
on a linear order. Let $X$ be a linear order.

\begin{definition}
    \label{elem_orbit}
    \lean{elem_orbit}\leanok
Let $f\colon X \to X$ an automorphism.
For any $x\in X$, the \textbf{orbit} of $x$ under $f$
is the set $\{f^n(x) : n \in \mathbb{Z}\}$.
\end{definition}

\begin{definition}
    \label{elem_orbital}
    \uses{elem_orbit}
    \lean{elem_orbital}\leanok
Let $f\colon X \to X$ an automorphism.
For any $x \in X$, the \textbf{orbital} of $x$ under $f$
is the convex closure of the orbit of $x$ under $f$.
\end{definition}

\begin{definition}
    \label{bump}
    \uses{elem_orbital}
    \lean{isBump}\leanok
A \textbf{bump} is an automorphism
$f\colon X \to X$ with exactly one non-singleton orbital.
\end{definition}

\begin{definition}
    \label{boundedBump}
    \uses{bump}
    \lean{isBoundedBump}\leanok
A \textbf{bounded bump} is a bump
whose unique non-singleton orbital
is either strictly bounded below
or strictly bounded above.
\end{definition}

\begin{definition}
    \label{bubbleR}
    \uses{boundedBump}
    \lean{bubbleR}\leanok
We define a relation $\sim_b$ on $X$ as follows:
$x \sim_b y$ if and only if there exists a
bounded bump such that $x$ and $y$ are in its
orbital or $x = y$.
\end{definition}

\begin{theorem}
    \label{refl_symm_bubbleR}
    \uses{bubbleR}
    \lean{reflexive_bubbleR, symmetric_bubbleR}\leanok
The relation $\sim_b$ is reflexive and symmetric.
\end{theorem}
\begin{proof}\leanok
The definition is clearly reflexive and the definition is symmetric in
$x$ and $y$.
\end{proof}

\begin{theorem}
    \label{convex_bubbleR}
    \uses{bubbleR}
    \lean{left_convex_bubbleR}\leanok
The relation $\sim_b$ is convex.
\end{theorem}
\begin{proof}\leanok
Let $x\in X$. We want to show that the $\sim_b$-equivalence class
of $x$ is an interval.
Let $y$ and $z$ be elements of $X$ such that
$x < y < z$ and $x \sim_b z$. It cannot be that $x = z$
and so we must have that $x,z \in O(f)$ for some bounded bump
$f\colon X \to X$. Since $O(f)$ is an interval and $x < y < z$,
we have that $y \in O(f)$. Thus, $x\sim_b y$ and so we are done.
\end{proof}

\begin{theorem}
    \label{trans_bubbleR}
    \uses{bubbleR}
The relation $\sim_b$ is transitive.
\end{theorem}
\begin{proof}
Let $x$, $y$, and $z$ be elements of $X$ such that
$x \sim_b y$ and $y \sim_b z$. If $x = y$ or $y =z$,
then we are done.
\end{proof}